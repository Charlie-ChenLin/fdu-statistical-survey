% 预调查内容(可直接 % 预调查内容(可直接 % 预调查内容(可直接 % 预调查内容(可直接 \input{presurvey_sections.tex} 插入主报告)
% 编译环境建议:ctexart / ctexrep(UTF-8)

\subsection{预调查一:LLM 问卷预调查(预测试)}
\label{sec:presurvey-llm}

\paragraph{动机与目的}
为提高正式调查的可执行性与测量质量,本研究在问卷投放前开展一轮 LLM 问卷预测试(pretest)。本问卷同时包含多段量表题与跳题结构(由“是否观看过微短剧”决定后续题块是否作答),仅依靠研究者人工通读,难以系统性暴露:题干理解分歧、选项覆盖不足、作答格式不一致以及跳题边界不清等问题。
因此,我们将 LLM 作为“可控的模拟受访者”,在给定大学生画像约束下逐题作答并形成结构化记录,以达成三项目的:一是检验问卷结构、题序与跳题逻辑的自洽性;二是观察题项表述在自然语言理解中的歧义点,提前修订措辞;三是预先跑通数据组织与清洗口径,为正式回收后的统计分析建立稳定的数据管道。该预测试仅用于\emph{发现问卷与流程问题},不用于替代真实样本的推断结论。

\paragraph{实施方法概述}
我们采用“访谈式逐题作答”的预测试流程:首先构建受访者画像(性别、年级、年龄、专业、学校),并将其作为作答一致性约束;随后制定统一作答规范(单选以字母表示、量表以 1--5 表示、多选以分隔符表示、开放题限制字数);最后按题号逐题提问,使同一名模拟受访者在同一会话中连续完成作答,以更贴近真实填答过程中的连贯性。预测试共生成 1000 份模拟答卷;在整理阶段参照预设规则剔除明显未完成文本(例如转写中包含“sorry”提示的记录)后,得到 974 份可用于质量检视的样本。

\begin{table}[htbp]
  \centering
  {%
  \small
  \setlength{\tabcolsep}{6pt}%
  \renewcommand{\arraystretch}{1.25}%
  \begin{tabular}{p{2.6cm} p{4.6cm} p{5.4cm} p{1.8cm}}
    \hline
    \textbf{环节} & \textbf{操作口径} & \textbf{目的与质量控制} & \textbf{产出} \\\hline
    画像设定 &
    生成大学生受访者画像(性别、年级、年龄、专业、学校)并作为“不可更改约束” &
    保证基本信息题与画像一致,便于检验问卷中“画像一致性”要求是否可落地 &
    画像字段(随记录保存) \\\hline
    作答规范 &
    为不同题型设定统一输出规范(单选/多选/量表/开放题)并限制输出冗余文本 &
    降低后续解析歧义,提前暴露题干或选项导致的“格式漂移”问题 &
    作答规范(提示词约束) \\\hline
    逐题预测试 &
    按题号逐题提问,同一名模拟受访者在同一上下文中连续作答,保留每题回答与转写记录 &
    检查题序、题干理解与上下文连贯性;对空答进行重试以减少缺失 &
    问答记录与转写 \\\hline
    跳题核验 &
    以“是否观看过微短剧”为分支条件,核验后续题块是否按预期出现/缺失 &
    重点排查跳题边界不清、题块遗漏或误作答,确保正式投放的逻辑一致 &
    题块缺失特征(由问答记录核验) \\\hline
    样本规模 &
    生成 1000 份模拟答卷,清洗后保留 974 份用于质量检视 &
    用“规模化预测试”提高问题暴露概率,并为后续问卷修订提供更充分的例证 &
    样本数量口径 \\\hline
    数据组织 &
    以“一名受访者一行”的结构化格式保存,并同步导出宽表用于快速检视 &
    便于快速定位异常题项(如缺失、非常规字符、选项越界)与进行描述性统计 &
    NDJSON/CSV \\\hline
    \hline
  \end{tabular}
  }%
  \caption{LLM 问卷预测试的流程口径与产出(见 \texttt{data/interviews.ndjson}、\texttt{data/interviews\_qa.csv})}
  \label{tab:presurvey-llm}
\end{table}

\paragraph{预测试输出的使用方式}
我们主要从两类信号评估问卷质量:其一是\emph{格式合规性}(单选/多选/量表是否满足输出约束、是否出现缺失或多余文本);其二是\emph{逻辑一致性}(基本信息是否与画像一致、是否触发正确的跳题、量表回答是否呈现合理波动)。
对发现的问题(如易混淆的题干或选项、开放题字数过长、跳题边界不清等),在正式发放前进行措辞与规则微调,并同步更新数据清洗脚本以保证一致性。


\subsection{预调查二:微博评论预调查(“短剧”“付费”相关语料)}
\label{sec:presurvey-weibo}

\paragraph{动机与目的}
为使问卷题项与选项更贴近受访者的真实语境,本研究补充开展微博评论语料预调查。相较于研究者主观推断,社交媒体评论能更直接呈现用户在“付费/会员/广告解锁/价格”等议题上的自然表达与争议点。我们围绕“短剧/微短剧”与“付费相关行为”两条主线获取评论语料,用于:提炼典型表述与高频关注点(例如价格敏感、广告时长容忍度、会员权益感知等),并将其反馈至正式问卷的措辞优化与选项补充,从而降低题项与真实语境脱节的风险。

\paragraph{数据采集与清洗流程}
我们以“博文---评论”为基本记录单元:先依据关键词检索相关博文,再抓取对应热评作为评论样本,并对文本进行基础清洗(去 HTML 标签并还原转义字符)。在抓取阶段,我们利用搜索关键词锁定“短剧/微短剧”语境,并默认启用“付费关键词过滤”(帖子正文或评论正文命中其一即可保留),以保证样本与“付费讨论”相关。项目当前示例数据覆盖近 180 天,共收集 2207 条“博文---评论”记录;在后续语料整理阶段,如需进一步聚焦“评论端的短剧$\times$付费讨论”,可在评论文本层面追加双关键词筛选,得到 49 条核心语料用于主题归纳与可视化呈现。

\begin{table}[htbp]
  \centering
  {%
  \small
  \setlength{\tabcolsep}{6pt}%
  \renewcommand{\arraystretch}{1.25}%
  \begin{tabular}{p{2.6cm} p{4.6cm} p{5.4cm} p{1.8cm}}
    \hline
    \textbf{环节} & \textbf{采集/筛选口径} & \textbf{说明与质量控制} & \textbf{产出} \\\hline
    检索范围 &
    以“短剧/微短剧”及付费相关组合词检索相关博文,限定近 180 天时间窗口 &
    兼顾召回与主题相关性;时间窗口用于控制语境的时效性与可比性 &
    候选博文 \\\hline
    评论单元 &
    抓取每条博文下的热评,将“1 条博文 + 1 条评论”作为记录单元 &
    以评论作为自然语言语料的主体,保留必要的上下文信息便于理解与去重 &
    评论样本 \\\hline
    付费相关性 &
    默认启用付费关键词过滤:帖子正文或评论正文命中其一即可保留(可选关闭) &
    在保证召回的同时提高主题相关性,避免大量“短剧但不涉及付费”的噪声样本 &
    过滤后样本 \\\hline
    二次聚焦(可选) &
    在评论文本层面追加“短剧词$\times$付费词”双关键词筛选,形成核心语料 &
    更聚焦于“评论端的付费讨论”,便于做主题归纳、举例与可视化呈现 &
    核心语料 \\\hline
    去噪与质控 &
    对文本进行去 HTML 与字符还原;可选剔除营销号/大号以降低推广内容占比 &
    降低广告/搬运内容对主题归纳的污染,提高语料可读性与解释性 &
    清洗文本 \\\hline
    结构化字段 &
    记录检索词、抓取时间、博文与评论文本、互动量、以及必要的用户摘要信息 &
    支持后续按时间/关键词/互动量进行分组统计,并便于复核数据来源 &
    NDJSON \\\hline
    预调查用途 &
    对核心语料进行主题归纳与高频词统计,辅助修订正式问卷的题项与选项 &
    将“自然语言证据”转化为可测量的题项表达,提升问卷的贴合度与可答性 &
    主题摘要/词频结果 \\\hline
    \hline
  \end{tabular}
  }%
  \caption{微博评论语料预调查的采集口径与使用方式(见 \texttt{outputs/weibo\_shortdrama\_comments\_180d.ndjson})}
  \label{tab:presurvey-weibo}
\end{table}

\paragraph{合规与伦理说明(简要)}
采集内容来自公开可见的社交媒体文本,研究目的为学术分析与问卷设计改进;在报告呈现中仅展示去标识化的聚合结果与匿名化示例,不呈现可用于识别个体的账号信息,且严格控制采集频率以避免对平台造成干扰。
 插入主报告)
% 编译环境建议:ctexart / ctexrep(UTF-8)

\subsection{预调查一:LLM 问卷预调查(预测试)}
\label{sec:presurvey-llm}

\paragraph{动机与目的}
为提高正式调查的可执行性与测量质量,本研究在问卷投放前开展一轮 LLM 问卷预测试(pretest)。本问卷同时包含多段量表题与跳题结构(由“是否观看过微短剧”决定后续题块是否作答),仅依靠研究者人工通读,难以系统性暴露:题干理解分歧、选项覆盖不足、作答格式不一致以及跳题边界不清等问题。
因此,我们将 LLM 作为“可控的模拟受访者”,在给定大学生画像约束下逐题作答并形成结构化记录,以达成三项目的:一是检验问卷结构、题序与跳题逻辑的自洽性;二是观察题项表述在自然语言理解中的歧义点,提前修订措辞;三是预先跑通数据组织与清洗口径,为正式回收后的统计分析建立稳定的数据管道。该预测试仅用于\emph{发现问卷与流程问题},不用于替代真实样本的推断结论。

\paragraph{实施方法概述}
我们采用“访谈式逐题作答”的预测试流程:首先构建受访者画像(性别、年级、年龄、专业、学校),并将其作为作答一致性约束;随后制定统一作答规范(单选以字母表示、量表以 1--5 表示、多选以分隔符表示、开放题限制字数);最后按题号逐题提问,使同一名模拟受访者在同一会话中连续完成作答,以更贴近真实填答过程中的连贯性。预测试共生成 1000 份模拟答卷;在整理阶段参照预设规则剔除明显未完成文本(例如转写中包含“sorry”提示的记录)后,得到 974 份可用于质量检视的样本。

\begin{table}[htbp]
  \centering
  {%
  \small
  \setlength{\tabcolsep}{6pt}%
  \renewcommand{\arraystretch}{1.25}%
  \begin{tabular}{p{2.6cm} p{4.6cm} p{5.4cm} p{1.8cm}}
    \hline
    \textbf{环节} & \textbf{操作口径} & \textbf{目的与质量控制} & \textbf{产出} \\\hline
    画像设定 &
    生成大学生受访者画像(性别、年级、年龄、专业、学校)并作为“不可更改约束” &
    保证基本信息题与画像一致,便于检验问卷中“画像一致性”要求是否可落地 &
    画像字段(随记录保存) \\\hline
    作答规范 &
    为不同题型设定统一输出规范(单选/多选/量表/开放题)并限制输出冗余文本 &
    降低后续解析歧义,提前暴露题干或选项导致的“格式漂移”问题 &
    作答规范(提示词约束) \\\hline
    逐题预测试 &
    按题号逐题提问,同一名模拟受访者在同一上下文中连续作答,保留每题回答与转写记录 &
    检查题序、题干理解与上下文连贯性;对空答进行重试以减少缺失 &
    问答记录与转写 \\\hline
    跳题核验 &
    以“是否观看过微短剧”为分支条件,核验后续题块是否按预期出现/缺失 &
    重点排查跳题边界不清、题块遗漏或误作答,确保正式投放的逻辑一致 &
    题块缺失特征(由问答记录核验) \\\hline
    样本规模 &
    生成 1000 份模拟答卷,清洗后保留 974 份用于质量检视 &
    用“规模化预测试”提高问题暴露概率,并为后续问卷修订提供更充分的例证 &
    样本数量口径 \\\hline
    数据组织 &
    以“一名受访者一行”的结构化格式保存,并同步导出宽表用于快速检视 &
    便于快速定位异常题项(如缺失、非常规字符、选项越界)与进行描述性统计 &
    NDJSON/CSV \\\hline
    \hline
  \end{tabular}
  }%
  \caption{LLM 问卷预测试的流程口径与产出(见 \texttt{data/interviews.ndjson}、\texttt{data/interviews\_qa.csv})}
  \label{tab:presurvey-llm}
\end{table}

\paragraph{预测试输出的使用方式}
我们主要从两类信号评估问卷质量:其一是\emph{格式合规性}(单选/多选/量表是否满足输出约束、是否出现缺失或多余文本);其二是\emph{逻辑一致性}(基本信息是否与画像一致、是否触发正确的跳题、量表回答是否呈现合理波动)。
对发现的问题(如易混淆的题干或选项、开放题字数过长、跳题边界不清等),在正式发放前进行措辞与规则微调,并同步更新数据清洗脚本以保证一致性。


\subsection{预调查二:微博评论预调查(“短剧”“付费”相关语料)}
\label{sec:presurvey-weibo}

\paragraph{动机与目的}
为使问卷题项与选项更贴近受访者的真实语境,本研究补充开展微博评论语料预调查。相较于研究者主观推断,社交媒体评论能更直接呈现用户在“付费/会员/广告解锁/价格”等议题上的自然表达与争议点。我们围绕“短剧/微短剧”与“付费相关行为”两条主线获取评论语料,用于:提炼典型表述与高频关注点(例如价格敏感、广告时长容忍度、会员权益感知等),并将其反馈至正式问卷的措辞优化与选项补充,从而降低题项与真实语境脱节的风险。

\paragraph{数据采集与清洗流程}
我们以“博文---评论”为基本记录单元:先依据关键词检索相关博文,再抓取对应热评作为评论样本,并对文本进行基础清洗(去 HTML 标签并还原转义字符)。在抓取阶段,我们利用搜索关键词锁定“短剧/微短剧”语境,并默认启用“付费关键词过滤”(帖子正文或评论正文命中其一即可保留),以保证样本与“付费讨论”相关。项目当前示例数据覆盖近 180 天,共收集 2207 条“博文---评论”记录;在后续语料整理阶段,如需进一步聚焦“评论端的短剧$\times$付费讨论”,可在评论文本层面追加双关键词筛选,得到 49 条核心语料用于主题归纳与可视化呈现。

\begin{table}[htbp]
  \centering
  {%
  \small
  \setlength{\tabcolsep}{6pt}%
  \renewcommand{\arraystretch}{1.25}%
  \begin{tabular}{p{2.6cm} p{4.6cm} p{5.4cm} p{1.8cm}}
    \hline
    \textbf{环节} & \textbf{采集/筛选口径} & \textbf{说明与质量控制} & \textbf{产出} \\\hline
    检索范围 &
    以“短剧/微短剧”及付费相关组合词检索相关博文,限定近 180 天时间窗口 &
    兼顾召回与主题相关性;时间窗口用于控制语境的时效性与可比性 &
    候选博文 \\\hline
    评论单元 &
    抓取每条博文下的热评,将“1 条博文 + 1 条评论”作为记录单元 &
    以评论作为自然语言语料的主体,保留必要的上下文信息便于理解与去重 &
    评论样本 \\\hline
    付费相关性 &
    默认启用付费关键词过滤:帖子正文或评论正文命中其一即可保留(可选关闭) &
    在保证召回的同时提高主题相关性,避免大量“短剧但不涉及付费”的噪声样本 &
    过滤后样本 \\\hline
    二次聚焦(可选) &
    在评论文本层面追加“短剧词$\times$付费词”双关键词筛选,形成核心语料 &
    更聚焦于“评论端的付费讨论”,便于做主题归纳、举例与可视化呈现 &
    核心语料 \\\hline
    去噪与质控 &
    对文本进行去 HTML 与字符还原;可选剔除营销号/大号以降低推广内容占比 &
    降低广告/搬运内容对主题归纳的污染,提高语料可读性与解释性 &
    清洗文本 \\\hline
    结构化字段 &
    记录检索词、抓取时间、博文与评论文本、互动量、以及必要的用户摘要信息 &
    支持后续按时间/关键词/互动量进行分组统计,并便于复核数据来源 &
    NDJSON \\\hline
    预调查用途 &
    对核心语料进行主题归纳与高频词统计,辅助修订正式问卷的题项与选项 &
    将“自然语言证据”转化为可测量的题项表达,提升问卷的贴合度与可答性 &
    主题摘要/词频结果 \\\hline
    \hline
  \end{tabular}
  }%
  \caption{微博评论语料预调查的采集口径与使用方式(见 \texttt{outputs/weibo\_shortdrama\_comments\_180d.ndjson})}
  \label{tab:presurvey-weibo}
\end{table}

\paragraph{合规与伦理说明(简要)}
采集内容来自公开可见的社交媒体文本,研究目的为学术分析与问卷设计改进;在报告呈现中仅展示去标识化的聚合结果与匿名化示例,不呈现可用于识别个体的账号信息,且严格控制采集频率以避免对平台造成干扰。
 插入主报告)
% 编译环境建议:ctexart / ctexrep(UTF-8)

\subsection{预调查一:LLM 问卷预调查(预测试)}
\label{sec:presurvey-llm}

\paragraph{动机与目的}
为提高正式调查的可执行性与测量质量,本研究在问卷投放前开展一轮 LLM 问卷预测试(pretest)。本问卷同时包含多段量表题与跳题结构(由“是否观看过微短剧”决定后续题块是否作答),仅依靠研究者人工通读,难以系统性暴露:题干理解分歧、选项覆盖不足、作答格式不一致以及跳题边界不清等问题。
因此,我们将 LLM 作为“可控的模拟受访者”,在给定大学生画像约束下逐题作答并形成结构化记录,以达成三项目的:一是检验问卷结构、题序与跳题逻辑的自洽性;二是观察题项表述在自然语言理解中的歧义点,提前修订措辞;三是预先跑通数据组织与清洗口径,为正式回收后的统计分析建立稳定的数据管道。该预测试仅用于\emph{发现问卷与流程问题},不用于替代真实样本的推断结论。

\paragraph{实施方法概述}
我们采用“访谈式逐题作答”的预测试流程:首先构建受访者画像(性别、年级、年龄、专业、学校),并将其作为作答一致性约束;随后制定统一作答规范(单选以字母表示、量表以 1--5 表示、多选以分隔符表示、开放题限制字数);最后按题号逐题提问,使同一名模拟受访者在同一会话中连续完成作答,以更贴近真实填答过程中的连贯性。预测试共生成 1000 份模拟答卷;在整理阶段参照预设规则剔除明显未完成文本(例如转写中包含“sorry”提示的记录)后,得到 974 份可用于质量检视的样本。

\begin{table}[htbp]
  \centering
  {%
  \small
  \setlength{\tabcolsep}{6pt}%
  \renewcommand{\arraystretch}{1.25}%
  \begin{tabular}{p{2.6cm} p{4.6cm} p{5.4cm} p{1.8cm}}
    \hline
    \textbf{环节} & \textbf{操作口径} & \textbf{目的与质量控制} & \textbf{产出} \\\hline
    画像设定 &
    生成大学生受访者画像(性别、年级、年龄、专业、学校)并作为“不可更改约束” &
    保证基本信息题与画像一致,便于检验问卷中“画像一致性”要求是否可落地 &
    画像字段(随记录保存) \\\hline
    作答规范 &
    为不同题型设定统一输出规范(单选/多选/量表/开放题)并限制输出冗余文本 &
    降低后续解析歧义,提前暴露题干或选项导致的“格式漂移”问题 &
    作答规范(提示词约束) \\\hline
    逐题预测试 &
    按题号逐题提问,同一名模拟受访者在同一上下文中连续作答,保留每题回答与转写记录 &
    检查题序、题干理解与上下文连贯性;对空答进行重试以减少缺失 &
    问答记录与转写 \\\hline
    跳题核验 &
    以“是否观看过微短剧”为分支条件,核验后续题块是否按预期出现/缺失 &
    重点排查跳题边界不清、题块遗漏或误作答,确保正式投放的逻辑一致 &
    题块缺失特征(由问答记录核验) \\\hline
    样本规模 &
    生成 1000 份模拟答卷,清洗后保留 974 份用于质量检视 &
    用“规模化预测试”提高问题暴露概率,并为后续问卷修订提供更充分的例证 &
    样本数量口径 \\\hline
    数据组织 &
    以“一名受访者一行”的结构化格式保存,并同步导出宽表用于快速检视 &
    便于快速定位异常题项(如缺失、非常规字符、选项越界)与进行描述性统计 &
    NDJSON/CSV \\\hline
    \hline
  \end{tabular}
  }%
  \caption{LLM 问卷预测试的流程口径与产出(见 \texttt{data/interviews.ndjson}、\texttt{data/interviews\_qa.csv})}
  \label{tab:presurvey-llm}
\end{table}

\paragraph{预测试输出的使用方式}
我们主要从两类信号评估问卷质量:其一是\emph{格式合规性}(单选/多选/量表是否满足输出约束、是否出现缺失或多余文本);其二是\emph{逻辑一致性}(基本信息是否与画像一致、是否触发正确的跳题、量表回答是否呈现合理波动)。
对发现的问题(如易混淆的题干或选项、开放题字数过长、跳题边界不清等),在正式发放前进行措辞与规则微调,并同步更新数据清洗脚本以保证一致性。


\subsection{预调查二:微博评论预调查(“短剧”“付费”相关语料)}
\label{sec:presurvey-weibo}

\paragraph{动机与目的}
为使问卷题项与选项更贴近受访者的真实语境,本研究补充开展微博评论语料预调查。相较于研究者主观推断,社交媒体评论能更直接呈现用户在“付费/会员/广告解锁/价格”等议题上的自然表达与争议点。我们围绕“短剧/微短剧”与“付费相关行为”两条主线获取评论语料,用于:提炼典型表述与高频关注点(例如价格敏感、广告时长容忍度、会员权益感知等),并将其反馈至正式问卷的措辞优化与选项补充,从而降低题项与真实语境脱节的风险。

\paragraph{数据采集与清洗流程}
我们以“博文---评论”为基本记录单元:先依据关键词检索相关博文,再抓取对应热评作为评论样本,并对文本进行基础清洗(去 HTML 标签并还原转义字符)。在抓取阶段,我们利用搜索关键词锁定“短剧/微短剧”语境,并默认启用“付费关键词过滤”(帖子正文或评论正文命中其一即可保留),以保证样本与“付费讨论”相关。项目当前示例数据覆盖近 180 天,共收集 2207 条“博文---评论”记录;在后续语料整理阶段,如需进一步聚焦“评论端的短剧$\times$付费讨论”,可在评论文本层面追加双关键词筛选,得到 49 条核心语料用于主题归纳与可视化呈现。

\begin{table}[htbp]
  \centering
  {%
  \small
  \setlength{\tabcolsep}{6pt}%
  \renewcommand{\arraystretch}{1.25}%
  \begin{tabular}{p{2.6cm} p{4.6cm} p{5.4cm} p{1.8cm}}
    \hline
    \textbf{环节} & \textbf{采集/筛选口径} & \textbf{说明与质量控制} & \textbf{产出} \\\hline
    检索范围 &
    以“短剧/微短剧”及付费相关组合词检索相关博文,限定近 180 天时间窗口 &
    兼顾召回与主题相关性;时间窗口用于控制语境的时效性与可比性 &
    候选博文 \\\hline
    评论单元 &
    抓取每条博文下的热评,将“1 条博文 + 1 条评论”作为记录单元 &
    以评论作为自然语言语料的主体,保留必要的上下文信息便于理解与去重 &
    评论样本 \\\hline
    付费相关性 &
    默认启用付费关键词过滤:帖子正文或评论正文命中其一即可保留(可选关闭) &
    在保证召回的同时提高主题相关性,避免大量“短剧但不涉及付费”的噪声样本 &
    过滤后样本 \\\hline
    二次聚焦(可选) &
    在评论文本层面追加“短剧词$\times$付费词”双关键词筛选,形成核心语料 &
    更聚焦于“评论端的付费讨论”,便于做主题归纳、举例与可视化呈现 &
    核心语料 \\\hline
    去噪与质控 &
    对文本进行去 HTML 与字符还原;可选剔除营销号/大号以降低推广内容占比 &
    降低广告/搬运内容对主题归纳的污染,提高语料可读性与解释性 &
    清洗文本 \\\hline
    结构化字段 &
    记录检索词、抓取时间、博文与评论文本、互动量、以及必要的用户摘要信息 &
    支持后续按时间/关键词/互动量进行分组统计,并便于复核数据来源 &
    NDJSON \\\hline
    预调查用途 &
    对核心语料进行主题归纳与高频词统计,辅助修订正式问卷的题项与选项 &
    将“自然语言证据”转化为可测量的题项表达,提升问卷的贴合度与可答性 &
    主题摘要/词频结果 \\\hline
    \hline
  \end{tabular}
  }%
  \caption{微博评论语料预调查的采集口径与使用方式(见 \texttt{outputs/weibo\_shortdrama\_comments\_180d.ndjson})}
  \label{tab:presurvey-weibo}
\end{table}

\paragraph{合规与伦理说明(简要)}
采集内容来自公开可见的社交媒体文本,研究目的为学术分析与问卷设计改进;在报告呈现中仅展示去标识化的聚合结果与匿名化示例,不呈现可用于识别个体的账号信息,且严格控制采集频率以避免对平台造成干扰。
 插入主报告)
% 编译环境建议:ctexart / ctexrep(UTF-8)

\subsection{预调查一:LLM 问卷预调查(预测试)}
\label{sec:presurvey-llm}

\paragraph{动机与目的}
为提高正式调查的可执行性与测量质量,本研究在问卷投放前开展一轮 LLM 问卷预测试(pretest)。本问卷同时包含多段量表题与跳题结构(由“是否观看过微短剧”决定后续题块是否作答),仅依靠研究者人工通读,难以系统性暴露:题干理解分歧、选项覆盖不足、作答格式不一致以及跳题边界不清等问题。
因此,我们将 LLM 作为“可控的模拟受访者”,在给定大学生画像约束下逐题作答并形成结构化记录,以达成三项目的:一是检验问卷结构、题序与跳题逻辑的自洽性;二是观察题项表述在自然语言理解中的歧义点,提前修订措辞;三是预先跑通数据组织与清洗口径,为正式回收后的统计分析建立稳定的数据管道。该预测试仅用于\emph{发现问卷与流程问题},不用于替代真实样本的推断结论。

\paragraph{实施方法概述}
我们采用“访谈式逐题作答”的预测试流程:首先构建受访者画像(性别、年级、年龄、专业、学校),并将其作为作答一致性约束;随后制定统一作答规范(单选以字母表示、量表以 1--5 表示、多选以分隔符表示、开放题限制字数);最后按题号逐题提问,使同一名模拟受访者在同一会话中连续完成作答,以更贴近真实填答过程中的连贯性。预测试共生成 1000 份模拟答卷;在整理阶段参照预设规则剔除明显未完成文本(例如转写中包含“sorry”提示的记录)后,得到 974 份可用于质量检视的样本。

\begin{table}[htbp]
  \centering
  {%
  \small
  \setlength{\tabcolsep}{6pt}%
  \renewcommand{\arraystretch}{1.25}%
  \begin{tabular}{p{2.6cm} p{4.6cm} p{5.4cm} p{1.8cm}}
    \hline
    \textbf{环节} & \textbf{操作口径} & \textbf{目的与质量控制} & \textbf{产出} \\\hline
    画像设定 &
    生成大学生受访者画像(性别、年级、年龄、专业、学校)并作为“不可更改约束” &
    保证基本信息题与画像一致,便于检验问卷中“画像一致性”要求是否可落地 &
    画像字段(随记录保存) \\\hline
    作答规范 &
    为不同题型设定统一输出规范(单选/多选/量表/开放题)并限制输出冗余文本 &
    降低后续解析歧义,提前暴露题干或选项导致的“格式漂移”问题 &
    作答规范(提示词约束) \\\hline
    逐题预测试 &
    按题号逐题提问,同一名模拟受访者在同一上下文中连续作答,保留每题回答与转写记录 &
    检查题序、题干理解与上下文连贯性;对空答进行重试以减少缺失 &
    问答记录与转写 \\\hline
    跳题核验 &
    以“是否观看过微短剧”为分支条件,核验后续题块是否按预期出现/缺失 &
    重点排查跳题边界不清、题块遗漏或误作答,确保正式投放的逻辑一致 &
    题块缺失特征(由问答记录核验) \\\hline
    样本规模 &
    生成 1000 份模拟答卷,清洗后保留 974 份用于质量检视 &
    用“规模化预测试”提高问题暴露概率,并为后续问卷修订提供更充分的例证 &
    样本数量口径 \\\hline
    数据组织 &
    以“一名受访者一行”的结构化格式保存,并同步导出宽表用于快速检视 &
    便于快速定位异常题项(如缺失、非常规字符、选项越界)与进行描述性统计 &
    NDJSON/CSV \\\hline
    \hline
  \end{tabular}
  }%
  \caption{LLM 问卷预测试的流程口径与产出(见 \texttt{data/interviews.ndjson}、\texttt{data/interviews\_qa.csv})}
  \label{tab:presurvey-llm}
\end{table}

\paragraph{预测试输出的使用方式}
我们主要从两类信号评估问卷质量:其一是\emph{格式合规性}(单选/多选/量表是否满足输出约束、是否出现缺失或多余文本);其二是\emph{逻辑一致性}(基本信息是否与画像一致、是否触发正确的跳题、量表回答是否呈现合理波动)。
对发现的问题(如易混淆的题干或选项、开放题字数过长、跳题边界不清等),在正式发放前进行措辞与规则微调,并同步更新数据清洗脚本以保证一致性。


\subsection{预调查二:微博评论预调查(“短剧”“付费”相关语料)}
\label{sec:presurvey-weibo}

\paragraph{动机与目的}
为使问卷题项与选项更贴近受访者的真实语境,本研究补充开展微博评论语料预调查。相较于研究者主观推断,社交媒体评论能更直接呈现用户在“付费/会员/广告解锁/价格”等议题上的自然表达与争议点。我们围绕“短剧/微短剧”与“付费相关行为”两条主线获取评论语料,用于:提炼典型表述与高频关注点(例如价格敏感、广告时长容忍度、会员权益感知等),并将其反馈至正式问卷的措辞优化与选项补充,从而降低题项与真实语境脱节的风险。

\paragraph{数据采集与清洗流程}
我们以“博文---评论”为基本记录单元:先依据关键词检索相关博文,再抓取对应热评作为评论样本,并对文本进行基础清洗(去 HTML 标签并还原转义字符)。在抓取阶段,我们利用搜索关键词锁定“短剧/微短剧”语境,并默认启用“付费关键词过滤”(帖子正文或评论正文命中其一即可保留),以保证样本与“付费讨论”相关。项目当前示例数据覆盖近 180 天,共收集 2207 条“博文---评论”记录;在后续语料整理阶段,如需进一步聚焦“评论端的短剧$\times$付费讨论”,可在评论文本层面追加双关键词筛选,得到 49 条核心语料用于主题归纳与可视化呈现。

\begin{table}[htbp]
  \centering
  {%
  \small
  \setlength{\tabcolsep}{6pt}%
  \renewcommand{\arraystretch}{1.25}%
  \begin{tabular}{p{2.6cm} p{4.6cm} p{5.4cm} p{1.8cm}}
    \hline
    \textbf{环节} & \textbf{采集/筛选口径} & \textbf{说明与质量控制} & \textbf{产出} \\\hline
    检索范围 &
    以“短剧/微短剧”及付费相关组合词检索相关博文,限定近 180 天时间窗口 &
    兼顾召回与主题相关性;时间窗口用于控制语境的时效性与可比性 &
    候选博文 \\\hline
    评论单元 &
    抓取每条博文下的热评,将“1 条博文 + 1 条评论”作为记录单元 &
    以评论作为自然语言语料的主体,保留必要的上下文信息便于理解与去重 &
    评论样本 \\\hline
    付费相关性 &
    默认启用付费关键词过滤:帖子正文或评论正文命中其一即可保留(可选关闭) &
    在保证召回的同时提高主题相关性,避免大量“短剧但不涉及付费”的噪声样本 &
    过滤后样本 \\\hline
    二次聚焦(可选) &
    在评论文本层面追加“短剧词$\times$付费词”双关键词筛选,形成核心语料 &
    更聚焦于“评论端的付费讨论”,便于做主题归纳、举例与可视化呈现 &
    核心语料 \\\hline
    去噪与质控 &
    对文本进行去 HTML 与字符还原;可选剔除营销号/大号以降低推广内容占比 &
    降低广告/搬运内容对主题归纳的污染,提高语料可读性与解释性 &
    清洗文本 \\\hline
    结构化字段 &
    记录检索词、抓取时间、博文与评论文本、互动量、以及必要的用户摘要信息 &
    支持后续按时间/关键词/互动量进行分组统计,并便于复核数据来源 &
    NDJSON \\\hline
    预调查用途 &
    对核心语料进行主题归纳与高频词统计,辅助修订正式问卷的题项与选项 &
    将“自然语言证据”转化为可测量的题项表达,提升问卷的贴合度与可答性 &
    主题摘要/词频结果 \\\hline
    \hline
  \end{tabular}
  }%
  \caption{微博评论语料预调查的采集口径与使用方式(见 \texttt{outputs/weibo\_shortdrama\_comments\_180d.ndjson})}
  \label{tab:presurvey-weibo}
\end{table}

\paragraph{合规与伦理说明(简要)}
采集内容来自公开可见的社交媒体文本,研究目的为学术分析与问卷设计改进;在报告呈现中仅展示去标识化的聚合结果与匿名化示例,不呈现可用于识别个体的账号信息,且严格控制采集频率以避免对平台造成干扰。
